\documentclass[12pt,australian]{article}
\usepackage{lingmacros}
\usepackage{tree-dvips}
\usepackage[utf8]{inputenc}
\usepackage{csquotes,xpatch}% recommended
\usepackage[australian]{babel}
\usepackage{amsmath}
\usepackage[backend=biber,<options>]{biblatex}
\usepackage{graphicx}
\addbibresource[location=local]{references.bib}


\title{Resource Use and the Value-Productivity Tradeoff in Australian Winegrowing Regions}
\author{Bryce Polley}
\date{20/06/2023}

\begin{document}
\begin{sloppypar}

\section{Introduction}
When assessing strategies for viticulture there is a trade-off between the amount of resources invested and the resultant quality and quantity of grapes produced \autocite{montalvo-falconSustainabilityResearchWine2023}. Due to recent economic, social and environmental pressures, such as: bush fires, tariffs, and shifting attitudes to alcohol consumption; the chosen emphasis taken in balancing the value-productivity tradeoff has become more important \autocite{toumbourouStudentSurveyTrends2018,wineaustraliaNationalVintageReport2020,wineaustraliaNationalVintageReport2022}. When considering pressures such as those prior listed, the integration of viticulture within the Australian wine industry is a compounding effect.

In this analysis we observed relationships between quantity and quality through the use of linear models. For the purpose of this study; quality was defined by the financial value of grapes per tonne. An extensive amount of research into a variety of factor's effect on grape quality and yield exists. Due to the lack of long-term and in-depth data sources, individual effects are often studied in isolation \autocite{abbalDecisionSupportSystem2016}. The lack of consolidated datasets also restricts the ability to gain statistical insight at a large scale across multiple regions \autocite{keithjonesAustralianWineIndustry2002,knightFirmResourcesDevelopment2019}. The dataset used for this analysis includes data collected for the past 10 years from a multitude of vineyards located over a diverse range of Australian winegrowing regions.

\section{Methods}
We created four linear models to explore the relationships between resource use and vineyard outputs (see table 1). The models looked at the value and quantity of grapes produced by vineyards, as a total and as a scale of area harvested. We compared values to observe how economies of scale affect the use of resources.

\begin{table}[]
    \begin{center}
      \caption{Summary of models; their predictors, covariates and variable interactions.}
      \label{tab:tab1}
      \begin{tabular}{ccccc}
        & Response & Predictors & Covariates & Interactions \\
        Model 1 & Yield & Water Used, Scope 1 Emissions & Area Harvested, Year, GI Region & N\textbackslash A \\
        Model 2 & ${\text{Yield}}\over{\text{Area Harvested}}$ & Water Used, Scope 1 Emissions & Area Harvested, Year, GI Region & Area Harvested * Scope 1 Emissions, Area Harvested * Water Use, Year * Region \\
        Model 3 & \small{$\text{Yield} {\times} \text{Average Sale Price}$} &  Water Used, Scope 1 Emissions& Area Harvested, Year, GI Region & Area Harvested * Scope 1 Emissions, Area Harvested * Water Use, Year * Region  \\
        Model 4 & ${\text{Yield} \times \text{Average Sale Price}}\over{\text{Area Harvested}}$ & Water Used, Scope 1 Emissions & Area Harvested, Year, GI Region  & N\textbackslash A
      \end{tabular}
    \end{center}
\end{table}

\subsection{Data}
The data used in this analysis was provided by Sustainable Winegrowing Australia, Australia's national wine industry sustainability program. Sustainable winegrowing Australia aims to facilitate grape-growers and winemakers in demonstrating and improving their sustainability \autocite{swaSustainableWingrowingAustralia2022}. Data recorded by Sustainable Winegrowing Australia is entered manually by winegrowers using a web based interface; with some fields being optional. Two subsets of this data were defined by vineyards that recorded values for average price of grape sales per tonne, and vineyards who did not. Both subsets contained: region, harvest year, yield, area harvested, water used and fuel used (in litres for diesel, petrol, biodiesel and LPG). To enable comparisons, total fuel was converted to amount of carbon emissions in metric tons.

The first subset of data was used for Model 1 and Model 2 (see Table 1). This subset contained 5298 samples spanning the period from 2012 to 2022, covering 57 GI Regions and 1432 separate vineyards.

The second subset of data, was limited to vineyards that recorded a value for their average sale price of grapes per tonne. This subset was used for Model 3 and Model 4 (see Table 1); and contained 2878 samples spanning the period from 2015 to 2022, covering 51 GI Regions and 944 separate vineyards.

Data was limited to samples that had recorded values for variables used (see Table 1). After reviewing correlation coefficients the data was logarithmically transformed, centred and scaled by standard deviation. Two values for average sale price were removed from the dataset, due to a recording of \$1. Unreported values for average prices per tonne were filled in using regional averages taken from Wine Australia's annual reports, where they were available \autocite{wineaustraliaNationalVintageReport2019,wineaustraliaNationalVintageReport2020,wineaustraliaNationalVintageReport2021,wineaustraliaNationalVintageReport2022,winemakersfederationofaustraliaNationalVintageReport2012,winemakersfederationofaustraliaNationalVintageReport2013,winemakersfederationofaustraliaNationalVintageReport2014,winemakersfederationofaustraliaNationalVintageReport2015,winemakersfederationofaustraliaNationalVintageReport2016,winemakersfederationofaustraliaNationalVintageReport2017,winemakersfederationofaustraliaNationalVintageReport2018}.

Other variables including the use of renewable energy, contractors; and pressures such as frost, fire and disease were also explored. Variables that did not significantly contribute to the prediction of a response variable were excluded.

\subsection{Total Emissions}
Emissions were calculated from the total diesel, petrol, bio-diesel and LPG used for irrigation and activities within the vineyard. The equations given from the Australian National Greenhouse Accounts Factors, shown as 

\begin{equation}
\label{(1)}
    tCO_{2}e={{Q \times EC \times EF1 + EF3}\over{1000}},
\end{equation}

was used to convert the quantity of fuel in litres, $Q$, using a prescribed Energy Content, $EC$, and emission factors of scope one, $EF1$, and scope three, $EF3$, to tonnes of Carbon Dioxide equivalent, $tCO2e$ \autocite{departmentofclimatechangeenergytheenvironmentandwaterAustralianNationalGreenhouse2022}.

The variables were reviewed for correlations by using a Pearson's Correlation Coefficient (see Tables 1, 2 and 3). This was undertaken for data on the original scale (see Table 1) and for data as a logarithmic transform (see Table 2). All P-values were found to be significant (< 2.200E-16), except the non-transformed values for water used (see Table 3). The logarithmic transforms performed the best due to a skew likely caused by a greater number of smaller vineyards within the dataset (see Table 4).

\subsection{Region}
The site of a vineyard predetermines several physical parameters such as climate, geology and soil; making location a widely considered key determinant of grape yield and quality \autocite{abbalDecisionSupportSystem2016,agostaRegionalClimateVariability2012,fragaMultivariateClusteringViticultural2017}. Differences in vineyard locations were captured through the use of Geographical Indicator Regions (GI Regions). Each GI Region has its own unique mixture of climatic and geophsical properties that describes a unique winegrowing region within Australia; these regions were predefined by Wine Australia \autocite{hallidayAustralianWineEncyclopedia2009,oliverReviewSoilPhysical2013,soarClimateDriversRed2008}.

The climatic properties of a GI Region are summarised in the Sustainable Winegrowing Australia user manual \autocite{swaSustainableWinegrowingAustralia2021}. The user manual describes climates by rainfall and temperature, creating supersets of Regions of similar climatic properties. The climatic groups were used to illustrate similarities and differences occurring in areas larger than GI regions.

\subsection{Analysis}
General Linear Models were used as they offered the ability to produce statistical models that were explicit in the relationships between predictors and response variables. They also allowed the exploration of interactions between predictors and easily comparable differences in the influence and magnitude of relationships.

Data preprocessing, such as logarithmic transforms, was done using the Python programming language \autocite{g.vanrossumPythonTutorialTechnical1995}. Linear models were created using the R statistical programming language \autocite{rcoreteamLanguageEnvironmentStatistical2021}. These models were created iteratively to explore a variety of variable interactions and approaches to modelling the data. Not all explored approaches yielded improvements or accurate models. Alternate approaches included the use of Splines, hierarchical regression, Additive and Generalised Linear Models. Other variables were also explored but not used due to low reporting values such as fertiliser, tractor and contractor use. The use of only scope one emissions was due to the same reason where scope 2 sources were recorded sporadically at best.

\subsection{Model Validation}
Models were validated using K-fold cross validation calculated through the R Caret Package \autocite{kuhnBuildingPredictiveModels2008}. K-fold cross validation works by removing a subset of data from the sample used to train models and then predicts those variables to determine how sensitive the model is to changes in the sample data. For this analysis each model was validated using 10 folds, repeated 100 times. % A reference for this is absolutely needed.

\section{Results}
\subsection{Exploratory Analysis}
Simple linear relationships between variables were explored using Pearson Correlation Coefficients. This was undertaken for data on the original scale (see Table 2) and for data as a logarithmic transform (see Table 3). Strong relationships were found to be present, as all P-values were considered significant (< 2.200E-16, see Tables 2 and 3), except for the non-transformed values for water used (see Table 4). The logarithmic transforms showed the strongest correlations, this was likely due to a skew caused by a greater number of smaller vineyards within the dataset (see Table 5).

\begin{table}[]
  \caption{Summary of models, their predictors, covariates and variable interactions.}
  \label{tab:tab2}
  \begin{tabular}{cccccccc}
  Variable                             & Yield      & Area       & Water Used & Scope One Emissions & $\text{Yield}\over\text{Area}$ & Average Price Per Tonne & ${\text{Average Price per tonne}\over\text{Area}}$ \\
  Yield                                & 1.000E+00  & 7.440E-01  & -4.309E-03 & 7.290E-01           & 3.500E-01            & -2.262E-01              & -1.644E-01                           \\
  Area                                 & 7.440E-01  & 1.000E+00  & -5.331E-03 & 8.921E-01           & 7.854E-02            & -1.178E-01              & -2.042E-01                           \\
  Water Used                           & -4.309E-03 & -5.331E-03 & 1.000E+00  & -1.929E-03          & -5.600E-03           & -3.562E-02              & -2.669E-02                           \\
  Scope One Emissions                  & 7.290E-01  & 8.921E-01  & -1.929E-03 & 1.000E+00           & 9.357E-02            & -9.422E-02              & -1.933E-01                           \\
  $\text{Yield}\over\text{Area}$                 & 3.500E-01  & 7.854E-02  & -5.600E-03 & 9.357E-02           & 1.000E+00            & -4.849E-01              & -1.698E-01                           \\
  Average Price Per Tonne              & -2.262E-01 & -1.178E-01 & -3.562E-02 & -9.422E-02          & -4.849E-01           & 1.000E+00               & 4.732E-01                            \\
  ${\text{Average Price per tonne}\over\text{Area}}$ & -1.644E-01 & -2.042E-01 & -2.669E-02 & -1.933E-01          & -1.698E-01           & 4.732E-01               & 1.000E+00                           
  \end{tabular}
  \end{table}

  \begin{table}[]
    \caption{Pearson correlation coefficients for each logarithmically transformed variable.}
    \label{tab:tab3}
    \begin{tabular}{clllllll}
    Variable                                           & \multicolumn{1}{c}{Yield} & \multicolumn{1}{c}{Area} & \multicolumn{1}{c}{Water Used} & \multicolumn{1}{c}{Scope One Emissions} & \multicolumn{1}{c}{$\text{Yield}\over\text{Area}$} & \multicolumn{1}{c}{Average Price Per Tonne} & \multicolumn{1}{c}{${\text{Average Price per tonne}\over\text{Area}}$} \\
    Yield                                              & 1.000E+00                 & 8.822E-01                & 8.245E-01                      & 7.617E-01                               & 9.353E-01                                          & -4.591E-01                                  & -8.918E-01                                                             \\
    Area                                               & 8.822E-01                 & 1.000E+00                & 7.750E-01                      & 8.311E-01                               & 6.742E-01                                          & -1.911E-01                                  & -8.474E-01                                                             \\
    Water Used                                         & 8.245E-01                 & 7.750E-01                & 1.000E+00                      & 6.668E-01                               & 7.292E-01                                          & -4.881E-01                                  & -8.300E-01                                                             \\
    Scope One Emissions                                & 7.617E-01                 & 8.311E-01                & 6.668E-01                      & 1.000E+00                               & 6.086E-01                                          & -1.559E-01                                  & -7.063E-01                                                             \\
    $\text{Yield}\over\text{Area}$                     & 9.353E-01                 & 6.742E-01                & 7.292E-01                      & 6.086E-01                               & 1.000E+00                                          & -5.625E-01                                  & -8.076E-01                                                             \\
    Average Price Per Tonne                            & -4.591E-01                & -1.911E-01               & -4.881E-01                     & -1.559E-01                              & -5.625E-01                                         & 1.000E+00                                   & 6.592E-01                                                              \\
    ${\text{Average Price per tonne}\over\text{Area}}$ & -8.918E-01                & -8.474E-01               & -8.300E-01                     & -7.063E-01                              & -8.076E-01                                         & 6.592E-01                                   & 1.000E+00                                                             
    \end{tabular}
    \end{table}

    \begin{table}[]
      \caption{P-values for the non-transformed water used variable's Pearson correlation coefficients.}
    \label{tab:tab4}
      \begin{tabular}{cl}
      Variable                                           & Water Used \\
      Yield                                              & 7.538E-01  \\
      Area                                               & 6.981E-01  \\
      Scope One Emissions                                & 8.883E-01  \\
      $\text{Yield}\over\text{Area}$                     & 6.836E-01  \\
      Average Price Per Tonne                            & 5.600E-02  \\
      ${\text{Average Price per tonne}\over\text{Area}}$ & 1.522E-01 
      \end{tabular}
      \end{table}

    \begin{table}[]
      \caption{Summary statistics for each variable on the original scale..}
      \label{tab:tab5}
      \begin{tabular}{clllllll}
      Variable &
        \multicolumn{1}{c}{Yield} &
        \multicolumn{1}{c}{Area} &
        \multicolumn{1}{c}{Water Used} &
        \multicolumn{1}{c}{Scope One Emissions} &
        \multicolumn{1}{c}{$\text{Yield}\over\text{Area}$} &
        \multicolumn{1}{c}{Average Price Per Tonne} &
        \multicolumn{1}{c}{${\text{Average Price per tonne}\over\text{Area}}$} \\
      Yield                                              & 1.000E+00  & 8.822E-01  & 8.245E-01  & 7.617E-01  & 9.353E-01  & -4.591E-01 & -8.918E-01 \\
      Area                                               & 8.822E-01  & 1.000E+00  & 7.750E-01  & 8.311E-01  & 6.742E-01  & -1.911E-01 & -8.474E-01 \\
      Water Used                                         & 8.245E-01  & 7.750E-01  & 1.000E+00  & 6.668E-01  & 7.292E-01  & -4.881E-01 & -8.300E-01 \\
      Scope One Emissions                                & 7.617E-01  & 8.311E-01  & 6.668E-01  & 1.000E+00  & 6.086E-01  & -1.559E-01 & -7.063E-01 \\
      $\text{Yield}\over\text{Area}$                     & 9.353E-01  & 6.742E-01  & 7.292E-01  & 6.086E-01  & 1.000E+00  & -5.625E-01 & -8.076E-01 \\
      Average Price Per Tonne                            & -4.591E-01 & -1.911E-01 & -4.881E-01 & -1.559E-01 & -5.625E-01 & 1.000E+00  & 6.592E-01  \\
      ${\text{Average Price per tonne}\over\text{Area}}$ & -8.918E-01 & -8.474E-01 & -8.300E-01 & -7.063E-01 & -8.076E-01 & 6.592E-01  & 1.000E+00 
      \end{tabular}
      \end{table}

\subsection{General Linear Models}
Models 1 and 2 showed significant relationships between each of the predictors and their response variable (see Tables 6 and 7). Variables in models 3 and 4 reported similar significance; except for scope 1 emissions (see Tables 8 and 9). Scope one emissions was included in all models to directly compare the response variables as ratios of vineyard size to raw values. Even though not significant within models 3 and 4, when using the Pearson Correlation Coefficients, scope one emissions was strongly correlated to every Model's response variable; this was especially so for Model 1 and 4 (Yeild and average price per tonne as a ratio to area harvested, respectively).

The comparison of models performance shows that the average price per tonne of grapes describes a great deal of the relationship between predictor and response when comparing model 2 to model 4 (see Table 10). This relationship between yield and average price was also illustrated in the correlation values between them (see Table 2).

\begin{table}[]
  \label{tab:tab6}
  \caption{Model 1 ANOVA summarising variable significance at the .5 level.}
  \begin{tabular}{llllll}
  Variable            & Df & Sum Sq    & Mean Sq   & F Value   & Pr(\textgreater{}F)    \\
  Year                & 9  & 7.060E+01 & 7.800E+00 & 8.353E+01 & \textless 2.20E-16 *** \\
  GI Region           & 54 & 1.507E+03 & 2.790E+01 & 2.972E+02 & \textless 2.20E-16 *** \\
  Area Harvested      & 1  & 3.211E+03 & 3.211E+03 & 3.419E+04 & \textless 2.20E-16 *** \\
  Water Used          & 1  & 1.040E+01 & 1.040E+01 & 1.103E+02 & \textless 2.20E-16 *** \\
  Scope One Emissions & 1  & 6.600E+00 & 6.600E+00 & 7.056E+01 & \textless 2.20E-16 ***
  \end{tabular}
\end{table}

\begin{table}[]
    \label{tab:tab7}
    \caption{Model 2 ANOVA summarising variable significance at the .5 level.}
    \begin{tabular}{llllll}
    Variable                    & Df  & Sum Sq    & Mean Sq   & F Value   & Pr(\textgreater{}F)    \\
    Area Harvested              & 1   & 2.407E+03 & 2.407E+03 & 1.080E+04 & \textless 2.20E-16 *** \\
    Scope One Emissions         & 1   & 3.989E+01 & 3.989E+01 & 1.789E+02 & \textless 2.20E-16 *** \\
    Water Used                  & 1   & 5.500E+02 & 5.500E+02 & 2.467E+03 & \textless 2.20E-16 *** \\
    Area Harvested*Scope One Emissions & 1 & 6.921E+01 & 6.921E+01 & 3.104E+02 & \textless 2.20E-16 *** \\
    Area Harvested * Water Used & 1   & 1.040E+00 & 1.040E+00 & 4.686E+00 & 3.045E-02 **           \\
    Year * GI Region            & 424 & 1.144E+03 & 2.700E+00 & 1.210E+01 & \textless 2.20E-16 ***
    \end{tabular}
\end{table}

\begin{table}[]
    \caption{Model 3 ANOVA summarising variable significance at the .5 level.}
    \label{tab:tab8}
    \begin{tabular}{llllll}
    Variable            & Df & Sum Sq    & Mean Sq   & F Value   & Pr(\textgreater{}F)    \\
    Year                & 6  & 1.324E+01 & 2.210E+00 & 8.748E+01 & \textless 2.20E-16 *** \\
    GI Region           & 50 & 6.498E+02 & 1.300E+01 & 5.151E+02 & \textless 2.20E-16 *** \\
    Area Harvested      & 1  & 2.142E+03 & 2.142E+03 & 8.491E+04 & \textless 2.20E-16 *** \\
    Water Used          & 1  & 3.200E-01 & 3.200E-01 & 1.259E+01 & 3.947E-04 **           \\
    Scope One Emissions & 1  & 4.000E-02 & 4.000E-02 & 1.492E+00 & 2.221E-01             
    \end{tabular}
\end{table}

\begin{table}[]
  \label{tab:tab9}
  \caption{Model 4 ANOVA summarising variable significance at the .5 level.}
  \begin{tabular}{llllll}
  Variable            & Df  & Sum Sq    & Mean Sq   & F Value   & Pr(\textgreater{}F)    \\
  Area Harvested      & 1   & 2.066E+03 & 2.066E+03 & 5.700E+04 & \textless 2.20E-16 *** \\
  Scope One Emissions & 1   & 6.000E-02 & 6.000E-02 & 1.569E+00 & 2.105E-01              \\
  Water Used          & 1   & 2.014E+02 & 2.014E+02 & 5.557E+03 & \textless 2.20E-16 *** \\
  Area Harvested*Scope One Emissions & 1 & 5.246E+01 & 5.246E+01 & 1.448E+03 & \textless 2.20E-16 *** \\
  Area Harvested * Water Used        & 1 & 7.270E+00 & 7.270E+00 & 2.005E+02 & \textless 2.20E-16 *** \\
  Year * GI Region    & 243 & 4.546E+02 & 1.870E+00 & 5.162E+01 & \textless 2.20E-16 ***
  \end{tabular}
\end{table}

\begin{table}[]
  \label{tab:tab 10}
  \caption{Comparison of Model Residuals}
  \begin{tabular}{llll}
        & Df   & Sum Sq    & Mean Sq   \\
  Model 1 & 5231 & 4.913E+02 & 1.000E-01 \\
  Model 2 & 4868 & 1.085E+03 & 2.200E-01 \\
  Model 3 & 2818 & 7.111E+01 & 3.000E-02 \\
  Model 4 & 2629 & 9.528E+01 & 4.000E-02
  \end{tabular}
\end{table}

\begin{table}[]
  \caption{Comparison of Model performance.}
  \label{tab:tab 11}
  \begin{tabular}{llllll}
          & RSE       & R2        & Adjusted 			R2 & F-statistic & P-Value           \\
  Model 1 & 3.065E-01 & 9.072E-01 & 9.061E-01      & 7.753E+02   & \textless 2.2e-16 \\
  Model 2 & 4.722E-01 & 7.951E-01 & 7.770E-01      & 4.403E+01   & \textless 2.2e-16 \\
  Model 3 & 1.589E-01 & 9.753E-01 & 9.748E-01      & 1.885E+03   & \textless 2.2e-16 \\
  Model 4 & 1.904E-01 & 9.669E-01 & 9.638E-01      & 3.095E+02   & \textless 2.2e-16
  \end{tabular}
\end{table}

Limitations included overestimating yield for models 1 and 2, (see Figures 1 and 2) and underestimating crop value in models 3 and 4 (see Figures 3 and 4).

 Reviewing the data to uncover reasons for this included the use of binary variables such as the utilisation of renewable energy, contractors, and the occurrence of disease, fire and frost; however none of these variables were able to explain why some vineyards produced less, or why other vineyards sold at higher prices than predicted. A wide variety of these influences were likely already explained within the use of year and GI Region, or the interaction of both variables. The change between some regions was dramatic, with particularly warmer and drier regions producing much higher volumes of grapes at lower prices (See Figures 5 and 6). The use of other variables and methods, specifically splines, were able to create a more normally distributed set of residuals but at a drastically reduced accuracy when comparing R2 and RSE. The introduction of known average prices per tonne also helped increase R2 values a small amount; it is important to not that it is common practice for wineries to purchase grapes at a regional average rate, likely resulting in much less variance within a region.

%\begin{figure}
%  \includegraphics[width=\linewidth]{figure1.jpg}
%  \caption{Model 1 residuals vs fitted (left) and QQ plot (right).}
%  \label{fig:fig1}
%\end{figure}

%\begin{figure}
%  \includegraphics[width=\linewidth]{figure2.jpg}
%  \caption{Model 2 residuals vs fitted (left) and QQ plot (right).}
%  \label{fig:fig2}
%\end{figure}

%\begin{figure}
%  \includegraphics[width=\linewidth]{figure3.jpg}
%  \caption{Model 3 residuals vs fitted (left) and QQ plot (right).}
%  \label{fig:fig3}
%\end{figure}

%\begin{figure}
%  \includegraphics[width=\linewidth]{figure4.jpg}
%  \caption{Model 4 residuals vs fitted (left) and QQ plot (right).}
%  \label{fig:fig4}
%\end{figure}

%\begin{figure}
%  \includegraphics[width=\linewidth]{figure5.jpg}
%  \caption{Average Yield per hectare for each GI Region.}
%  \label{fig:fig5}
%\end{figure}

%\begin{figure}
%  \includegraphics[width=\linewidth]{figure6.jpg}
%  \caption{Average Price Per Tonne for each GI Region.}
%  \label{fig:fig6}
%\end{figure}



% This belongs in the discussion
The relationship between scope one emissions and the response variables that included average sales price

It is possible that the relationships between scope one emissions and the response variables were closely tied to a vineyards area. This possibility could be explained through the emissions 

Noting that irrigation systems use fuel
and that the application of water was a significant variable in each model
scope one emissions' lack of significance and contribution given its F-statistics (See Tables 7 and 8),
 indicated that it is possible other vineyard activities requiring fuel are not as determining factors for a vineyards grape quality.


\printbibliography{}
\end{sloppypar}
\end{document}
