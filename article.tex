% Template for PLoS
% Version 3.6 Aug 2022
%
% % % % % % % % % % % % % % % % % % % % % %
%
% -- IMPORTANT NOTE
%
% This template contains comments intended 
% to minimize problems and delays during our production 
% process. Please follow the template instructions
% whenever possible.
%
% % % % % % % % % % % % % % % % % % % % % % % 
%
% Once your paper is accepted for publication, 
% PLEASE REMOVE ALL TRACKED CHANGES in this file 
% and leave only the final text of your manuscript. 
% PLOS recommends the use of latexdiff to track changes during review, as this will help to maintain a clean tex file.
% Visit https://www.ctan.org/pkg/latexdiff?lang=en for info or contact us at latex@plos.org.
%
%
% There are no restrictions on package use within the LaTeX files except that no packages listed in the template may be deleted.
%
% Please do not include colors or graphics in the text.
%
% The manuscript LaTeX source should be contained within a single file (do not use \input, \externaldocument, or similar commands).
%
% % % % % % % % % % % % % % % % % % % % % % %
%
% -- FIGURES AND TABLES
%
% Please include tables/figure captions directly after the paragraph where they are first cited in the text.
%
% DO NOT INCLUDE GRAPHICS IN YOUR MANUSCRIPT
% - Figures should be uploaded separately from your manuscript file. 
% - Figures generated using LaTeX should be extracted and removed from the PDF before submission. 
% - Figures containing multiple panels/subfigures must be combined into one image file before submission.
% For figure citations, please use "Fig" instead of "Figure".
% See http://journals.plos.org/plosone/s/figures for PLOS figure guidelines.
%
% Tables should be cell-based and may not contain:
% - spacing/line breaks within cells to alter layout or alignment
% - do not nest tabular environments (no tabular environments within tabular environments)
% - no graphics or colored text (cell background color/shading OK)
% See http://journals.plos.org/plosone/s/tables for table guidelines.
%
% For tables that exceed the width of the text column, use the adjustwidth environment as illustrated in the example table in text below.
%
% % % % % % % % % % % % % % % % % % % % % % % %
%
% -- EQUATIONS, MATH SYMBOLS, SUBSCRIPTS, AND SUPERSCRIPTS
%
% IMPORTANT
% Below are a few tips to help format your equations and other special characters according to our specifications. For more tips to help reduce the possibility of formatting errors during conversion, please see our LaTeX guidelines at http://journals.plos.org/plosone/s/latex
%
% For inline equations, please be sure to include all portions of an equation in the math environment.  For example, x$^2$ is incorrect; this should be formatted as $x^2$ (or $\mathrm{x}^2$ if the romanized font is desired).
%
% Do not include text that is not math in the math environment. For example, CO2 should be written as CO\textsubscript{2} instead of CO$_2$.
%
% Please add line breaks to long display equations when possible in order to fit size of the column. 
%
% For inline equations, please do not include punctuation (commas, etc) within the math environment unless this is part of the equation.
%
% When adding superscript or subscripts outside of brackets/braces, please group using {}.  For example, change "[U(D,E,\gamma)]^2" to "{[U(D,E,\gamma)]}^2". 
%
% Do not use \cal for caligraphic font.  Instead, use \mathcal{}
%
% % % % % % % % % % % % % % % % % % % % % % % % 
%
% Please contact latex@plos.org with any questions.
%
% % % % % % % % % % % % % % % % % % % % % % % %

\documentclass[10pt,letterpaper]{article}
\usepackage[top=0.85in,left=2.75in,footskip=0.75in]{geometry}

% amsmath and amssymb packages, useful for mathematical formulas and symbols
\usepackage{amsmath,amssymb}

% Use adjustwidth environment to exceed column width (see example table in text)
\usepackage{changepage}

% textcomp package and marvosym package for additional characters
\usepackage{textcomp,marvosym}

% cite package, to clean up citations in the main text. Do not remove.
\usepackage{cite}

% Use nameref to cite supporting information files (see Supporting Information section for more info)
\usepackage{nameref,hyperref}

% line numbers
\usepackage[right]{lineno}

% ligatures disabled
\usepackage[nopatch=eqnum]{microtype}
\DisableLigatures[f]{encoding = *, family = * }

% color can be used to apply background shading to table cells only
\usepackage[table]{xcolor}

% array package and thick rules for tables

% create "+" rule type for thick vertical lines
\newcolumntype{+}{!{\vrule width 2pt}}

% create \thickcline for thick horizontal lines of variable length
\newlength\savedwidth
\newcommand\thickcline[1]{%
  \noalign{\global\savedwidth\arrayrulewidth\global\arrayrulewidth 2pt}%
  \cline{#1}%
  \noalign{\vskip\arrayrulewidth}%
  \noalign{\global\arrayrulewidth\savedwidth}%
}

% \thickhline command for thick horizontal lines that span the table
\newcommand\thickhline{\noalign{\global\savedwidth\arrayrulewidth\global\arrayrulewidth 2pt}%
\hline
\noalign{\global\arrayrulewidth\savedwidth}}


% Remove comment for double spacing
%\usepackage{setspace} 
%\doublespacing

% Text layout
\raggedright
\setlength{\parindent}{0.5cm}
\textwidth 5.25in 
\textheight 8.75in

% Bold the 'Figure #' in the caption and separate it from the title/caption with a period
% Captions will be left justified
\usepackage[aboveskip=1pt,labelfont=bf,labelsep=period,justification=raggedright,singlelinecheck=off]{caption}
\renewcommand{\figurename}{Fig}

% Use the PLoS provided BiBTeX style
\bibliographystyle{plos2015}

% Remove brackets from numbering in List of References
\makeatletter
\renewcommand{\@biblabel}[1]{\quad#1.}
\makeatother



% Header and Footer with logo
\usepackage{lastpage,fancyhdr,graphicx}
\usepackage{epstopdf}
%\pagestyle{myheadings}
\pagestyle{fancy}
\fancyhf{}
%\setlength{\headheight}{27.023pt}
%\lhead{\includegraphics[width=2.0in]{PLOS-submission.eps}}
\rfoot{\thepage/\pageref{LastPage}}
\renewcommand{\headrulewidth}{0pt}
\renewcommand{\footrule}{\hrule height 2pt \vspace{2mm}}
\fancyheadoffset[L]{2.25in}
\fancyfootoffset[L]{2.25in}
\lfoot{\today}

%% Include all macros below

% \newcommand{\lorem}{{\bf LOREM}}
% \newcommand{\ipsum}{{\bf IPSUM}}

%% END MACROS SECTION

% Original draft packages
\usepackage{array}
\usepackage{amssymb}
\usepackage{adjustbox}
\usepackage{tabulary, ragged2e}
\usepackage{booktabs}
\usepackage{multirow}
\usepackage{csvsimple}
\usepackage{longtable}

\begin{document}


\vspace*{0.2in}

% Title must be 250 characters or less.
\begin{flushleft}
{\Large
\textbf\newline{The influence of resource use on yield versus sale price trade-off in Australian vineyards}
% \title{The influence of resource use on yield versus sale price trade-off in Australian vineyards}
% Please use "sentence case" for title and headings (capitalize only the first word in a title (or heading), the first word in a subtitle (or subheading), and any proper nouns).
}
\newline
% Insert author names, affiliations and corresponding author email (do not include titles, positions, or degrees).
\\
Bryce Boyd\textsuperscript{1,2,3,*},
Kate Helmstedt\textsuperscript{1},
Mardi Longbottom\textsuperscript{2},
Kerrie Mengersen\textsuperscript{1}
% {\textpilcrow}
\\
\bigskip
\textbf{1} Affiliation Faculty of Mathematical Sciences, Queensland University of Technology, Brisbane, QLD, Australia
\\
\textbf{2} Affiliation Australian Wine Research Institute, Adelaide, SA, Australia
\\
\textbf{3} Affiliation Food Agility CRC, Sydney, NSW, Australia
\\
\bigskip

% Insert additional author notes using the symbols described below. Insert symbol callouts after author names as necessary.
% 
% Remove or comment out the author notes below if they aren't used.
%


% % Primary Equal Contribution Note
% \Yinyang These authors contributed equally to this work.

% % Additional Equal Contribution Note
% % Also use this double-dagger symbol for special authorship notes, such as senior authorship.
% \ddag These authors also contributed equally to this work.

% % Current address notes
% \textcurrency Current Address: Dept/Program/Center, Institution Name, City, State, Country % change symbol to "\textcurrency a" if more than one current address note
% % \textcurrency b Insert second current address 
% % \textcurrency c Insert third current address

% % Deceased author note
% \dag Deceased

% % Group/Consortium Author Note
% \textpilcrow Membership list can be found in the Acknowledgments section.

% % Use the asterisk to denote corresponding authorship and provide email address in note below.
* Bryce.polley@awri.com.au

\end{flushleft}

\section*{Abstract}
% The word limit needs to be checked.

Strategies for achieving sustainability in the winegrowing industry requires balancing resource investment against the economic outcomes of resultant yields and sale price of the produce. Although there has been much research into the forecasting of outcomes and resource use, little has been done to illustrate the effects they have on one another and the consequential economic outcomes. In this analysis we observe relationships between resource use, yield and sale price through the use of statistical models. The dataset used for this analysis includes data collected for the past 10 years from 1261 vineyards located over a diverse range of Australian winegrowing regions. Yield and sale price were evaluated with respect to the resource use factors water use and Green House Gas (GHG) emissions. The analysis confirmed a strong relationship between area and resource use, with the overall area of a vineyard and its access to resources greatly determining the upper limit of yield. However, area was also negatively related to the average sale price of grapes; we find that higher average sale prices were connected to high resource inputs per area, rather than to the overall expenditure of resources. Regional and temporal effects on vineyard yield and average sales price were also identified. Overall, the analysis highlighted the importance of considering a vineyard's business goal, region, external pressures and economies of scale, when considering whether to pursue higher yields or higher average sales prices.

% \begin{highlights}
% \item Found that higher resource inputs per area are associated with higher sale prices, rather than overall resource expenditure.
% \item Identified significant differences in yield and sale prices across regions, highlighting the influence of localized conditions on vineyard profitability.
% \item Developed baseline models that demonstrate the critical relationship between vineyard area and both resource usage and economic outcomes.
% \item Analysis of long-term data offering a comprehensive view of industry trends highlighting resource use trends and economic outcomes.
% \end{highlights}

%%%%%%%%%%%%%%%%%%%%%%%%%%%%%%%%%%%%%%%%%%
%%                main text             %%
%%%%%%%%%%%%%%%%%%%%%%%%%%%%%%%%%%%%%%%%%%
\linenumbers

\section*{Introduction}

The focus on sustainability in agronomic industries has changed the way in which these enterprises do business. The diverse nature of vineyard operations results in complicated interactions with both environmental and economic outcomes. The interactions between different factors can put an environmentally advantageous decision at odds with an economic one, and conversely economic decisions can be detrimental to the environment~\cite{beaumelleBiodiversityConservationEcosystem2023}. As a result many studies are including more diverse ranges of data and scales. However, studies tend to focus on specific aspects, such as reducing chemical use or improving soil health, often focusing on singular relationships or variable outcomes~\cite{barriguinhaVineyardYieldEstimation2021,heFruitYieldPrediction2022,laurentReviewIssuesMethods2021}. Other analyses, look at life cycles or cost benefit analyses, reflecting a specific management decision's outcome~\cite{abbottAWRIAssessingEnvironmental2016,beauchetInterannualVariabilityEnvironmental2019,ferraraLifeCycleAssessment2018} or their cost benefits by utilising different scenarios to assess specific practices~\cite{fincoCombiningPrecisionViticulture2022a,singhBibliometricReviewUse2022}. It has been noted that some studies have relied too heavily on single points of engagement when relaying the idea of sustainability and vineyard outcomes, especially when the inclusion of more variables or data could have better described a situation as a whole~\cite{baianoOverviewSustainabilityWine2021,santiago-brownSustainabilityAssessmentWineGrape2015}. The changing nature of sustainability also calls for measures to be more than simple snapshots of data~\cite{baianoOverviewSustainabilityWine2021,montalvo-falconSustainabilityResearchWine2023,santiago-brownSustainabilityAssessmentWineGrape2015}. Given recent studies there is still growing interest from the winegrowing industry in further studies, with several recent reviews calling for more research to demonstrate the potential trade-offs between different operational approaches across different climates and regions~\cite{baianoOverviewSustainabilityWine2021,marianiSustainableWinegrowingCurrent2015,montalvo-falconSustainabilityResearchWine2023}.

A dilemma exists across agriculture through shared fundamental considerations of resource use such as water and fuel, and the resultant yield and crop value that are produced~\cite{hemmingCherryTomatoProduction2020,kawasakiQualityMattersMore2016, zhuEffectsNitrogenLevel2017}. The average price of grapes for wine production is driven through its integration within the wine industry. An important connection between grapes and their price is the grape's perceived quality, which initially defines a wine's potential through the grape's chemical makeup~\cite{blackTerpenoidsTheirRole2015,schreierFlavorCompositionWines1979}. Grapes of higher perceived quality or grapes from well known regions are likely to have higher prices~\cite{wineaustraliaNationalVintageReport2021}.  Grape quality is connected to the market value of winegrapes, with Wine Australia explicitly defining grape quality through the use of discrete price brackets in their annual reports~\cite{winemakersfederationofaustraliaNationalVintageReport2018}. It is important to note that the generalisation made to reflect quality through using average price assumes a due diligence of those who purchase the grapes~\cite{yeggeInfluenceSensoryNonsensory2001}. The economic sustainability of a vineyard is tied to this market culture, driven by the wine industry. The consideration of sustainability within viticulture is also subject to environmental and sociodemographic pressures~\cite{santiago-brownSustainabilityAssessmentWineGrape2015}. In the Australian context, these pressures include biosecurity, climate and international market changes~\cite{canadellMultidecadalIncreaseForest2021,longbottomRoleVineyardPractices2015,oliverReviewSoilPhysical2013}.
\par
There is an extensive amount of research into the varied effects of factors on grape quality and yield~\cite{heFruitYieldPrediction2022,laurentLocalInfluenceClimate2022,liakosMachineLearningAgriculture2018}. Due to the lack of long-term and in-depth data there is a lack of research on grape sale price and its driving factors. Furthermore, individual factors are often studied in isolation with yield and sales price not appearing together~\cite{abbalDecisionSupportSystem2016}. The lack of consolidated datasets restricts the ability to gain statistical insights at large scales and across multiple regions. As a result, broader studies are lacking~\cite{keithjonesAustralianWineIndustry2002,knightFirmResourcesDevelopment2019}. The dataset used for this analysis includes data spanning 10 years from a multitude of vineyards located over a diverse range of Australian winegrowing regions. We use this dataset to describe the relationship of resources related to water and fuel use with the output yield and average sale price of the resultant product, taking into account the size and location of the vineyard. In a practical sense, this aim is a baseline for comparison: given a vineyard within Australia, one could estimate the comparative efficiency with regard to the trade-off between invested resources, yield and average sale price. This is the first time that such a trade-off has been confirmed explicitly across such varying regions, scales and climates in the Australian winegrowing industry.
\section*{Methods}
\subsection*{Data}

\begin{table}[htb]
\caption{Summary of models; their predictors, covariates and variable interactions.}\label{tab:tab1}
  \resizebox{\textwidth}{!}{\begin{tabular}{@{}ccccc@{}}
    \toprule
    & \textbf{Response} & \textbf{Predictors} & \textbf{Covariates} & \textbf{Interactions} \\ \midrule
    \textbf{Model 1} & Yield & \begin{tabular}[c]{@{}c@{}}Water Used\\ scope one Emissions\end{tabular} & \begin{tabular}[c]{@{}c@{}}Area Harvested\\ Year\\ GI Region\end{tabular} & N/A \\
    \multicolumn{1}{l}{} & \multicolumn{1}{l}{} & \multicolumn{1}{l}{} & \multicolumn{1}{l}{} & \multicolumn{1}{l}{} \\
    \textbf{Model 2} & ${ \textrm{Yield}}\over{ \textrm{Area Harvested}}$ & \begin{tabular}[c]{@{}c@{}}Water Used\\ scope one Emissions\end{tabular} & \begin{tabular}[c]{@{}c@{}}Area Harvested\\ Year\\ GI Region\end{tabular} & \begin{tabular}[c]{@{}c@{}}Area Harvested * scope one Emissions\\ Area Harvested * Water Use\\ Year * Region\end{tabular} \\
    \multicolumn{1}{l}{} & \multicolumn{1}{l}{} & \multicolumn{1}{l}{} & \multicolumn{1}{l}{} & \multicolumn{1}{l}{} \\
    \textbf{Model 3} & {$\textrm{Yield} {\times} \textrm{Average Sale Price}$} & \begin{tabular}[c]{@{}c@{}}Water Used\\ Scope One Emissions\end{tabular} & \begin{tabular}[c]{@{}c@{}}Area Harvested\\ Year\\ GI Region\end{tabular} & N/A\\
    \multicolumn{1}{l}{} & \multicolumn{1}{l}{} & \multicolumn{1}{l}{} & \multicolumn{1}{l}{} & \multicolumn{1}{l}{} \\
    \textbf{Model 4} & Average Sale Price & \begin{tabular}[c]{@{}c@{}}Water Used\\ Scope One Emissions\end{tabular} & \begin{tabular}[c]{@{}c@{}}Area Harvested\\ Year\\ GI Region\end{tabular} & \begin{tabular}[c]{@{}c@{}}Area Harvested * Scope One Emissions\\ Area Harvested * Water Use\\ Year * Region\end{tabular}
        \\
        \multicolumn{1}{l}{} & \multicolumn{1}{l}{} & \multicolumn{1}{l}{} & \multicolumn{1}{l}{} & \multicolumn{1}{l}{} \\
     \textbf{Model 5} &  Average Sale Price & \begin{tabular}[c]{@{}c@{}}Water Used\\ Scope One Emissions\end{tabular} & \begin{tabular}[c]{@{}c@{}} Year\\ GI Region\end{tabular} & \begin{tabular}[c]{@{}c@{}} Year * Region\end{tabular}
        \end{tabular}}
\end{table}

Data used in these analyses were obtained from Sustainable Winegrowing Australia and Wine Australia. Sustainable Winegrowing Australia is Australia's national wine industry sustainability program, which aims to facilitate grape  growers and winemakers in demonstrating and improving their sustainability~\cite{swaSustainableWingrowingAustralia2022}. Wine Australia is an Australian Government statutory authority governed by the Wine Australia Act~\cite{attorney-generalsdepartmentWineAustraliaCorporation2010}. Data collected by Wine Australia are publicly available. The response (predicted) variables in this analysis were yield, defined as the total tonnes of grapes harvested, and average sale price of grapes in Australian dollars per tonne. Both response variables were examined as totals and also relative to area harvested (see Table~\ref{tab:tab1}). Values were compared in this manner to observe how economies of scale affect the use of resources.
\par
Data recorded by Sustainable Winegrowing Australia were entered manually by winegrowers voluntarily using a web based interface. For each model, vineyards were only included if they recorded all the variables used for the corresponding model (see Table~\ref{tab:tab1}). Each vineyard had at least recorded region, harvest year, yield and area harvested. Other variables that were used but not present for every single vineyard were average sale price, water used and fuel used (diesel, petrol, biodiesel and LPG). To enable direct comparisons between fuels, fuel use was converted to tonnes of carbon dioxide equivalent and collectively referenced to as emissions. All variables were continuous except for harvest year and region, which were categorical variables (Table~\ref{tab:tab1}).
\par
As data from Sustainable Wine Australia were voluntarily given, missing values were improved using regional average prices from the Wine Australia data. Data obtained from Wine Australia were collected via phone surveys and included: total tonnes purchased, average price per tonne and yearly change in price for region and grape varietal; with the data being publicly available.
\par
The dataset was split into two distinct subsets that could be used across the different models. The first subset contained all vineyards and was used for two models (Model 1 and Model 2, see Table~\ref{tab:tab1}). The second subset contained vineyards which either recorded a value for average price of sale per tonne through Sustainable Winegrowing Australia, or were within a region with an average price of sale recorded by Wine Australia; this subset was used for three further models (Models 3, 4 and 5, see Table~\ref{tab:tab1}). These subsets meant that the data would be limited to samples which had recorded values for the response variables (see Table~\ref{tab:tab1}), as every sample/vineyard had a recorded value for yield but not average price of sale per tonne.
\par
The first subset of data (used for Model 1 and Model 2, see Table~\ref{tab:tab1}) contained 5298 samples spanning the period from 2012 to 2022, covering 55 GI Regions and 1261 discrete vineyards.
\par
The second subset of data (used for Model 3, Model 4 and Model 5, see Table~\ref{tab:tab1}) contained 2878 samples spanning the period from 2015 to 2022, covering 51 GI Regions and 944 separate vineyards. Average price of sale per tonne was extracted from both Wine Australia (1842 values) and Sustainable Winegrowing Australia (remaining 1036 values).
\par
Additional variables were considered for analysis but were excluded due to being either underreported or had insignificant contributions to model accuracies. Variables explored but not used due to low reporting values included fertiliser (kg), electricity (Kw/h), and scope two emissions (TCO$^2$ equivalent). Variables considered but ultimately removed due to a lack of significant contributions to models, included the use of renewable energy, use of contractors, and pressures such as frost, fire and disease.
\par
Data preprocessing was conducted prior to analysis using the Python programming language~\cite{g.vanrossumPythonTutorialTechnical1995}. Preprocessing included the conversion of fuel to scope one emissions and prior calculations for all continuous variables which included logarithmic transformations, centring and scaling by standard deviation. 
We converted multiple emission sources into scope one emissions using the equation given from the Australian National Greenhouse Accounts Factors~\cite{agdeeNationalGreenhouseAccounts2021}. The calculation conducted using
\begin{equation}
\label{(1)}
    tCO_{2}e={{Q \times EC \times EF1 + EF3}\over{1000}},
\end{equation}
where emissions was the the product of the quantity of fuel in litres, $Q$, a prescribed Energy Content, $EC$, and emission factors (as given by the Australian National Greenhouse Accounts factors) of scope one, $EF1$, and scope three, $EF3$, to tonnes of Carbon Dioxide Emission equivalent, $tCO2e$~\cite{departmentofclimatechangeenergytheenvironmentandwaterAustralianNationalGreenhouse2022}.
\par
\begin{figure}[htb]
  % \resizebox{\textwidth}{!}{
  % \includegraphics{fig1.pdf}}
  \caption{Map of each GI Regions' (outlined by Wine Australia) average income for a vineyard of that region (average grape sale price per tonne $\times$ total tonnes yielded).}\label{fig:map}
\end{figure}%%

\par
Differences in vineyard locations were captured through the use of Geographical Indicator Regions defined by Wine Australia~\cite{hallidayAustralianWineEncyclopedia2009,oliverReviewSoilPhysical2013,soarClimateDriversRed2008}. Although vineyards generally differed in income, there were pronounced differences between regions as shown in Figure~\ref{fig:map}. These differences were likely due to the site of a vineyard being important as it predetermines several physical parameters such as climate, geology and soil, making location a widely considered key determinant of grape yield and average sale price~\cite{abbalDecisionSupportSystem2016,agostaRegionalClimateVariability2012,fragaMultivariateClusteringViticultural2017}. Each GI Region has its own unique mixture of climatic and geophysical properties that describes a unique winegrowing region within Australia and is a protected trademark under the Wine Australia act~\cite{attorney-generalsdepartmentWineAustraliaCorporation2010}. Both Wine Australia and Sustainable Winegrowing Australia used the same GI Region categorical variable format to describe location.
\par
 The climatic properties of each GI Region were summarised by using predefined classifications as per the~\cite{sustainablewinegrowingaustraliaSustainableWinegrowingAustralia2021} user manual. The user manual describes climates by rainfall and temperature, creating supersets of Regions of similar climatic properties. The climatic groups were used to illustrate similarities and differences occurring in areas larger than GI Regions.
These classifications were literal descriptors used by industry members to discern between different weather types within winegrowing regions. Other climatic descriptors were explored, such as the  Köppen climate classification, hoiwever this did not offer a granular enough index to highlight any trends or clustering within the data.
\par
\subsection*{Analysis}
Pairwise Pearson Correlation Coefficients (PPCC) were calculated to assess the potential existence of linear relationships between the input and predicted variables. To determine if a coefficient was indicative of a strong relationship, both the magnitude of the value and the associated confidence intervals were evaluated. P-values reflected the significance of a given correlation coefficient with statistical significance being declared when the corresponding P-value was lower than 0.05. PPCC analyses were undertaken for data on the original scale and for data as a logarithmic transform. Transforming data prior to calculating the coefficients changes several things. The logarithmic transform of the data alters the interpretation of the coefficients to percentage change; a coefficient will be indicative of the change in percentage of one variable compared to the other, scaling by standard deviation also changes this interpretation to be a percentage of that variables standard deviation. When considering the logarithmically transformed variables, a coefficient of 1 would indicate that the change of one variable by one percentage of its standard deviation would correlate to the other variable changing by one percent of its own standard deviation. The importance of this is the dimensionless nature of these relationships and that it can be translated directly to any vineyard's case that has a well known distribution.
\par
Five general linear models (GLMs) were created (see Table~\ref{tab:tab1}). GLMs were chosen as they offer the ability to produce statistical models that are explicit in the relationships between predictors and response variables.  General Linear Models also allowed the exploration of interactions between predictors and allow for easily comparable differences in the influence and magnitude of relationships. Model fit was measured in terms of $R^2$ and adjusted $R^2$ as well as F statistics, and t-tests were used to determine if predictors significantly contributed to the models when accounting for other variables, showing which specific years and areas contributed significantly. Both the PPCC and GLM analyses were created using the R statistical programming language~\cite{rcoreteamLanguageEnvironmentStatistical2021} with the Caret package~\cite{kuhnBuildingPredictiveModels2008}. 
\par
A variety of alternate methods were also explored, including splines, hierarchical regression, general additive models, and generalised linear models. These alternative approaches were not used as final models due to offering little further insights or improvements in accuracy.
\par
\subsection*{Model Validation}
Models were validated using K-fold cross validation. This was performed by removing a subset of data from the sample used to train models and then predicting those variables to determine how sensitive the model is to changes in the sample data. For this analysis, each model was validated using 10 folds, repeated 100 times.
\par
\section*{Results}
\subsection*{Summary Statistics}\label{sec:exp_anal}
 Table~\ref{tab:summary} shows the summary statistics of each variable in its original units. The range of these values shows the level of difference between some vineyards, with operations differing by orders of magnitude in size, yield and average price of sale (See Table~\ref{tab:tab1}).
\par
\begin{table}[htb]
  \caption{Summary statistics of each continuous variable.}\label{tab:summary}
      % \resizebox{\textwidth}{!}{
        \begin{tabular}{@{}cllll@{}}
          \toprule
          \textbf{Variable} & \textbf{Mean} & \textbf{\begin{tabular}[c]{@{}l@{}}Standard\\ Deviation\end{tabular}} & \textbf{Minimum} & \textbf{Maximum} \\ \midrule
          {Yield (tonnes)} & 7.76E+02 & 2.18E+03 & 1.00E+00 & 7.23E+04 \\
          \multicolumn{1}{l}{} &  &  &  &  \\
          {Area Harvested (ha)} & 6.67E+01 & 1.34E+02 & 7.00E-02 & 2.44E+03 \\
          \multicolumn{1}{l}{} &  &  &  &  \\
          {Water Used (ML)} & 7.47E+06 & 5.65E+08 & 1.00E+00 & 4.27E+10 \\
          \multicolumn{1}{l}{} &  &  &  &  \\
          {Scope One Emissions ($tCO_{2}e$)} & 4.17E+04 & 8.57E+04 & 6.76E+00 & 2.11E+06 \\
          \multicolumn{1}{l}{} &  &  &  &  \\
          $\mbox{Yield (tonnes)}\over \mbox{Area harvested (ha)}$ & 1.01E+01 & 8.13E+00 & 4.00E-02 & 8.63E+01 \\
          \multicolumn{1}{l}{} &  &  &  &  \\
          {Average Sale Price (AUD/tonne)} & 1.48E+03 & 9.22E+02 & 1.60E+02 & 2.60E+04 \\
          \multicolumn{1}{l}{} &  &  &  &  \\
          $\mbox{Average Sale Price (AUD/tonne)} \over \mbox{Area Harvested (ha)}$ & 1.35E+02 & 5.71E+02 & 1.75E-01 & 2.98E+04 \\ \bottomrule
          \end{tabular}
  % }
  \end{table}

PPCCs values for the transformed, centred and scaled variables are shown in Table~\ref{tab:pearson_correlation}. The process of centring and scaling variables changes them so that their mean is centred on 0 and all values are scaled to the variables standard deviation, such that 1 is equal to the variables standard deviation. All correlations were found to be statistically significant (p $<$ 2.200E-16), and except for Average Sale Price all variables were positively correlated. With water use, area harvested and emissions being positively correlated to yield, it can be considered that more resources and area are likely to lead to greater yields. The negative correlations between Average Sale Price and Yield, Water Use, Area and Scope One Emissions each indicated that Area Harvested and Fuel Use separately were not the determining factor for average sale price. The negative correlations are associative, not causal relationships (i.e using more water does not cause lower sale prices).
\par
\begin{table}[htb]
  \caption{Pairwise Pearson correlation coefficients for logarithmically transformed values.}\label{tab:pearson_correlation}
      \resizebox{\textwidth}{!}{
      \begin{tabular}{@{}cccccccc@{}}
          \toprule
          \textbf{} & \multicolumn{1}{c}{\textbf{Yield}} & \multicolumn{1}{c}{\textbf{Area Harvested}} & \multicolumn{1}{c}{\textbf{Water Used}} & \multicolumn{1}{c}{\textbf{\begin{tabular}[c]{@{}c@{}}Scope One\\ Emissions\end{tabular}}} & \multicolumn{1}{c}{\textbf{\begin{tabular}[c]{@{}c@{}}Yield by\\ Area\end{tabular}}} & \multicolumn{1}{c}{\textbf{Average Price}} & \multicolumn{1}{c}{\textbf{\begin{tabular}[c]{@{}c@{}}Average Price\\ by Area\end{tabular}}} \\ \midrule
          Yield & 1.00 & 0.88 & 0.82 & 0.76 & 0.96 & -0.46 & -0.88 \\
          Area Harvested & 0.88 & 1.00 & 0.78 & 0.83 & 0.73 & -0.19 & -0.81 \\
          Water Used & 0.82 & 0.78 & 1.00 & 0.67 & 0.76 & -0.49 & -0.82 \\
          \begin{tabular}[c]{@{}c@{}}Scope One\\ Emissions\end{tabular} & 0.76 & 0.83 & 0.67 & 1.00 & 0.65 & -0.16 & -0.67 \\
          \begin{tabular}[c]{@{}c@{}}Yield by\\ Area\end{tabular} & 0.96 & 0.73 & 0.76 & 0.65 & 1.00 & -0.54 & -0.84 \\
          Average Price & -0.46 & -0.19 & -0.49 & -0.16 & -0.54 & 1.00 & 0.72 \\
          \begin{tabular}[c]{@{}c@{}}Average Price\\ by Area\end{tabular} & -0.88 & -0.81 & -0.82 & -0.67 & -0.84 & 0.72 & 1.00 \\ \bottomrule
          \end{tabular}}
\end{table}

\subsection*{General Linear Models}

Each model had a high $R^2$ value, indicating that most of the variance within the data was described by the models (see Table~\ref{tab:modelsummary}) and statistical significance of F-tests (p $<$ 2.200E-16). Aside from 3 variables, all regression parameters were also statistically significant from zero, F-tests across each model's variables were (p $<$ 0.05). The three exceptions were: scope one emissions in Model 3 (p=0.22) and Model 4 (p=0.0.39), and the interaction between area harvested and water used in model 2 (p=0.22). Note that scope one emissions was included in all models to directly compare the response variables as ratios of vineyard size to raw values and because it was strongly correlated to the response variable in every model (except model 5) especially for Models 1 and 4 (Table~\ref{tab:pearson_correlation}). 
%A full list of regression coefficients 95\% CIs and p-values for each of the four models is provided in the appendix. % You gotta add these to the appendix in a more readable fashion
\par
\begin{table}[htb]
  \caption{Summary of models; their performance, F-statistics and Residual error.}\label{tab:modelsummary}
      \resizebox{\textwidth}{!}{
        \begin{tabular}{@{}cccccccc@{}}
          \toprule
           & $\mathbf{R^2}$ & \textbf{\begin{tabular}[c]{@{}c@{}} Adjusted\\ $\mathbf{R^2}$ \end{tabular}} & \textbf{F-Statistic} & \textbf{P-Value} & \textbf{\begin{tabular}[c]{@{}c@{}}Residual\\ Standard Error\end{tabular}} & \textbf{\begin{tabular}[c]{@{}c@{}}Residual Sum\\ of Squares\end{tabular}} & \textbf{\begin{tabular}[c]{@{}c@{}}Residual Mean\\ of Squares\end{tabular}} \\ \midrule
          \textbf{\begin{tabular}[c]{@{}c@{}}Model 1\\ \end{tabular}} & 0.9072 & 0.9061 & 775.3 & 2.200e-16 & 0.3065 & 491.3 & 0.1 \\
          \multicolumn{1}{l}{} & \multicolumn{1}{l}{} & \multicolumn{1}{l}{} & \multicolumn{1}{l}{} & \multicolumn{1}{l}{} & \multicolumn{1}{l}{} & \multicolumn{1}{l}{} & \multicolumn{1}{l}{} \\
          \textbf{\begin{tabular}[c]{@{}c@{}}Model 2\\ \end{tabular}} & 0.8291 & 0.8141 & 55.07 & 2.200e-16 & 0.4312
           & 905.03 & 0.19 \\
          \multicolumn{1}{l}{} & \multicolumn{1}{l}{} & \multicolumn{1}{l}{} & \multicolumn{1}{l}{} & \multicolumn{1}{l}{} & \multicolumn{1}{l}{} & \multicolumn{1}{l}{} & \multicolumn{1}{l}{} \\
          \textbf{\begin{tabular}[c]{@{}c@{}}Model 3\\ \end{tabular}} &0.9753 &0.9748 & 1885 & 2.200e-16 & 0.1589 & 71.11 & 0.03 \\
          \multicolumn{1}{l}{} & \multicolumn{1}{l}{} & \multicolumn{1}{l}{} & \multicolumn{1}{l}{} & \multicolumn{1}{l}{} & \multicolumn{1}{l}{} & \multicolumn{1}{l}{} & \multicolumn{1}{l}{} \\
          \textbf{\begin{tabular}[c]{@{}c@{}}Model 4\\ \end{tabular}} & 0.9091 & 0.9006 & 106.1 & 2.200e-16 & 0.3153  & 261.41 & 0.10 \\
          \multicolumn{1}{l}{} & \multicolumn{1}{l}{} & \multicolumn{1}{l}{} & \multicolumn{1}{l}{} & \multicolumn{1}{l}{} & \multicolumn{1}{l}{} & \multicolumn{1}{l}{} & \multicolumn{1}{l}{} \\
          \textbf{\begin{tabular}[c]{@{}c@{}}Model 5\\ \end{tabular}} & 0.9089 & 0.9004 & 107.2 & 2.200e-16 & 0.3155 & 262.04 & 0.10 \\ \bottomrule
          \end{tabular}
        }
\end{table}

Coefficients related to continuous variables are summarised in Table~\ref{tab:contvar}. In Model 1 all coefficients except for the intercept significantly contributed to the model (p $<$ 0.05), and in Model 2 all coefficients were statistically significant (p $<$ 0.05). However, scope one emissions did not significantly contribute to models 3, 4 and 5. Model 4 only had statistically significant contributions from the intercept and water use. Although the coefficient for water use was statistically significant for each model, it did not have the highest value; instead, area harvested, being an order of magnitude greater, dominated the models. Model 5 achieved a similar $R^2$ to Model 4 without area harvested, having stronger influences from water use and scope one emissions.

% Please add the following required packages to your document preamble:
% \usepackage{booktabs}
% \usepackage{multirow}
\begin{table}[htb]
  \caption{Summary of each Model's coefficients for continuous variables}\label{tab:contvar}
  \resizebox{\textwidth}{!}{
  \begin{tabular}{@{}cccccccc@{}}
  \toprule
  \textbf{} & \textbf{} & \textbf{Intercept} & \textbf{Area Harvested} & \textbf{Water Used} & \textbf{\begin{tabular}[c]{@{}c@{}}Scope One\\ Emissions\end{tabular}} & \textbf{\begin{tabular}[c]{@{}c@{}}Area Harvested \\ ** \\ Scope One Emissions\end{tabular}} & \textbf{\begin{tabular}[c]{@{}c@{}}Area Harvested \\ ** \\ Water Used\end{tabular}} \\ \midrule
  \multirow{2}{*}{Model 1} & Coefficient & -3.32E-02 & 7.42E-01 & 8.66E-02 & 6.73E-02 &  &  \\
   & Std Error & 1.96E-02 & 1.00E-02 & 8.90E-03 & 8.00E-03 &  &  \\
   &  &  &  &  &  &  &  \\
  \multirow{2}{*}{Model 2} & Coefficient & 1.70E-01 & 5.77E-01 & 1.08E-01 & 8.50E-02 & -4.97E-02 & -5.35E-02 \\
   & Std Error & 5.91E-02 & 1.48E-02 & 1.31E-02 & 1.17E-02 & 8.10E-03 & 8.40E-03 \\
   &  &  &  &  &  &  &  \\
  \multirow{2}{*}{Model 3} & Coefficient & 1.81E-02 & 9.71E-01 & -2.31E-02 & -7.00E-03 &  &  \\
   & Std Error & 1.30E-02 & 7.20E-03 & 6.90E-03 & 5.70E-03 &  &  \\
   &  &  &  &  &  &  &  \\
  \multirow{2}{*}{Model 4} & Coefficient & 1.45E-01 & 2.40E-03 & -4.66E-02 & -1.70E-02 & 1.15E-02 & 1.40E-03 \\
   & Std Error & 5.28E-02 & 1.50E-02 & 1.43E-02 & 1.18E-02 & 7.90E-03 & 8.30E-03 \\
   &  &  &  &  &  &  &  \\
  \multirow{2}{*}{Model 5} & Coefficient & 1.52E-01 &  & -4.04E-02 & -1.71E-02 &  &  \\
   & Std Error & 5.27E-02 &  & 1.13E-02 & 9.70E-03 &  &  \\ \cmidrule(l){2-8} 
  \end{tabular}
  }
  \end{table}

The regression coefficients for the Year variable of each model is depicted in Figure~\ref{fig:yearly}. For each model, the first year for a model's was used as the baseline: 2012 for Models 1 and 2, and 2015 for Models 3, 4 and 5. Adelaide Hills is used as the regional baseline with the interaction between year and region using the first year and the Adelaide Hills as the baseline. Region and year contributed, in some but not all cases, more than the other variables. However, the coefficients varied substantially over the years, and were not significantly different from zero in some years. Models 4 and 5 are very similar in Figure~\ref{fig:yearly}, indicating that the exclusion of area does not greatly affect the contribution from yearly influence. Models 4 and 5 have the most prominent trends, showing an increase in yearly effects over time, with Model 3 also increasing from 2016 to 2018, but plateauing afterwards. Model 1 and 2 do not show a clear trend, but do drop during 2017 and 2018 after increasing in the first 3 years.
\par
\begin{figure}[htb]
  % \includegraphics{fig2.pdf}
  \caption{Model Coefficient values for Year, with standard error bars. The coefficiens represent the influence that each year had on the model outcomes (see Table 1 for model outcomes and their predictors). Each years influence is comparative to the first year in the models, 2012.}t\label{fig:yearly}
  \end{figure}

% Search for where you mention quality and where you mention value as this does become misleading!!!
%
% This is partly because of the intercept
%
% It depends what else is in the model.
%
% This direct comparisons are 
%
% The correlated between input vars needs to be discussed earlier.
%
%difference between the continuous variables coefficients is the change from positive to negative values. This change occurs between the Models for Yield (Model 1 and 2) and the Models for value (Models 3 and 4); where all but the coefficient for area harvested had the opposite sign (see Table~\ref{tab:contvar}). These models also differ in an order of magnitude when looking at resource use, with the coefficients for yield being smaller than those for value. 
%
% The resource input is different by an order of magnitude between value and yield
%
% Area has the strongest nfluence
%
% The models as a ratio of area show opposite trends - this could be indicative that smaller vineyards tend twoards more valuable crops and larger to crops of greater yield. 
%

Regional differences by temperature and rainfall are summarised in Figure~\ref{fig:spider}. The most notable difference is between vineyards within 'Hot' and 'Very Dry' regions (warm inland regions), where high average sale prices are historically low, and yield is high. Water Use changes dramatically between these regions as well, with water being a driving force for yield but not necessarily average sale price. The warmer and drier regions tend to also have larger vineyards. These regional differences are further shown in Figure~\ref{fig:yield_vs_value_area}, as a ratio of vineyard; where there is a pronounced difference in 'Hot' and 'Very Dry' regions producing more per area with lower average sale prices per area than other regions.
\par
\begin{figure}[htb]
  % \resizebox{\textwidth}{!}{
  % \includegraphics{fig3.pdf}}
  \caption{Radar plot of climatic profile's resource use, yield and average sale price. Each point shows the overal contribution of a variable to the vineyards outcomes for that climate. The left reflects vineyards in different climatic temperatures. The right reflects vineyards in different rainfall climates.}\label{fig:spider}
\end{figure}

Figure~\ref{fig:yield_vs_value_area} further shows the emphasis that 'Hot' areas have on high yields, with low average sale price compared with other regions. Scaling average price and yield by area shows a strong negative trend, trading quantity for higher sales prices.
\par
\begin{figure}[htb]
  % \resizebox{\textwidth}{!}{
  % \includegraphics{fig4.pdf}}
    \caption{Scatter plot of vineyard yield against the average sale price as ratios to area harvested. The axes are in standard deviations with points coloured by climate.}\label{fig:yield_vs_value_area}
\end{figure}

Table~\ref{tab:kfold} shows the validation results of each of the models. The $R^2$ measures of fit show similar results to the initial models, with a slight decrease as expected, indicating that the models are robust and consistent.
%
%
% At least summary statistics of the baseline model.
%
%
\par
\begin{table}[htb]
  \caption{Model validation using k-fold cross validation, for 10 folds repeated 100 times.}\label{tab:kfold}
  % \resizebox{\textwidth}{!}{
  \begin{tabular}{@{}cccc@{}}
    \toprule
    \textbf{} & \textbf{\begin{tabular}[c]{@{}c@{}}Residual Mean\\ Squared Error\end{tabular}} & \textbf{R2} & \textbf{\begin{tabular}[c]{@{}c@{}}Mean Average \\ Error\end{tabular}} \\ \midrule
    \textbf{Model 1} & .309 & .905 & .2165 \\
    \textbf{Model 2} & .457 & .7921 & .313 \\
    \textbf{Model 3} & .165 & .972 & .101 \\
    \textbf{Model 4} & .348 & .878 & .182 \\ 
    \textbf{Model 5} & .348 & .878 & .183 \\ \bottomrule
    \end{tabular}
  % }
  \end{table}

\section*{Discussion}

There was an expected strong relationship between size and resource use, with the overall area of a vineyard and its access to resources greatly determining the upper limit of potential yield. However, size was also inversely related to the potential average sale price. Higher average sales prices were also related to high resource inputs per area instead of the overall expenditure of resources. Vineyard yields and sales price changed greatly by region and year. Even given regional and yearly changes, there was a strong connection between smaller vineyards and higher sales prices. This could have been due to more attention available when managing smaller properties.
\par
The lack of significance of scope one emissions and its contribution to models, given its F-statistics, could be indicative that other vineyard activities requiring fuel are not leading factors for a vineyards average sale price. The relationship between yield, value and area was not simply about efficiently producing the most grapes. It is possible that the relationship of scope one emissions between yield and sale price was closely tied to a vineyard's area due to requiring more fuel to address more issues over greater distances. It is difficult to discern the connection of scope one emissions directly, as fuel can be used for a broad category of activities. 
\par
There are important considerations unique to winegrowing compared to other agricultural industries. The vertical integration of winegrowing within the wine industry ties winegrowers to secondary and tertiary industries, such as wine production, packaging, transport and sales. This results in unique issues and considerations for each vineyard, where on-the-ground decisions are influenced by other wine industry's choices, such as the use of sustainable practices in vineyards as a requirement for sale in overseas markets; notably these interactions can be further complicated by some winegrowers being completely integrated into a wine company, while others are not~\cite{knightFirmResourcesDevelopment2019}. Incorporating decisions into the model could help describe the contributing factors to regional differences beyond resource consumption, motivating the call for more granular data and more sophisticated modelling.
\par
There are many on-the-ground decisions that influence both sales price and yield. The decision to prioritise average sale price over quantity, is governed by complex physical and social forces, for example international market demands, disease pressures and natural disasters~\cite{abadCoverCropsViticulture2021,cortezUsingDataMining2009,hallWithinseasonTemporalVariation2011,i.goodwinManagingSoilWater2009,kasimatiPredictingGrapeSugar2022,oliverReviewSoilPhysical2013,srivastavaNondestructiveSensingMethods2018}, with many of these occurrences being highlighted throughout the reports from Wine Australia over the past decade~\cite{wineaustraliaNationalVintageReport2019,wineaustraliaNationalVintageReport2021,wineaustraliaNationalVintageReport2022,winemakersfederationofaustraliaNationalVintageReport2013,winemakersfederationofaustraliaNationalVintageReport2014,winemakersfederationofaustraliaNationalVintageReport2015,winemakersfederationofaustraliaNationalVintageReport2016,winemakersfederationofaustraliaNationalVintageReport2017,winemakersfederationofaustraliaNationalVintageReport2018}. However, the changes in the coefficients (see Figure~\ref{fig:yearly}) are not reflective of many known occurrences, such as the 2020 bush fires, which had higher values for coefficients than prior years; During the 2020 bush fires 40,000 tonnes of grapes were lost across 18 different wine regions due to bush fires and smoke taint. In comparison to countrywide pressures such as drought, this damage made up only 3\% of the total amount of grapes for that year; although acknowledged as a considerable loss on an individual basis, it was deemed to be only a minor national concern by Wine Australia when compared to other environmental pressures such as drought~\cite{wineaustraliaNationalVintageReport2020}
\par
Climatic pressures are an important consideration for growers, especially those in warmer and drier regions. The Wine Australia reports also show that warm inland regions have seen a decline in profit over the past decade, whereas regions with lower average sales prices did not~\cite{wineaustraliaNationalVintageReport2019,wineaustraliaNationalVintageReport2020,wineaustraliaNationalVintageReport2021,winemakersfederationofaustraliaNationalVintageReport2013,winemakersfederationofaustraliaNationalVintageReport2014,winemakersfederationofaustraliaNationalVintageReport2015,winemakersfederationofaustraliaNationalVintageReport2016,winemakersfederationofaustraliaNationalVintageReport2017,winemakersfederationofaustraliaNationalVintageReport2018}. Vineyards in warm inland regions also tend to be larger, making up for lower sale prices with larger yields. Considering the negative correlation of average price to area, for this strategy to work, economies of scale become an important factor. Given the large quantities of grapes that can be produced by some vineyards, even at low margins there is the potential to be profitable. However, the increasing climatic pressures mixed with the requirement for larger volumes of water, make the sustainability of some vineyards come into question. Furthermore, intensive farming in general is known to jeopardise the sustainability of an operation through the degradation of soil and waterways~\cite{capelloEffectsTractorPasses2019,linHydropedologySynergisticIntegration2012,pisciottaGroundwaterNitrateRisk2015}. There are established methods that can help to mitigate these effects, such as the use of cover crops, midrow crop rotation and efficient irrigation.
\par
Some regions appeared to produce grapes of lower average sale price at scale whilst others achieved higher priced grapes in lower volumes. This empirical finding is consistent with Wine Australia's annual reports, which shows that some GI regions are known for producing large amounts of lower grade (low value per tonne) grapes~\cite{wineaustraliaNationalVintageReport2022,winemakersfederationofaustraliaNationalVintageReport2017}. Comparatively, other regions only produce grapes of higher sales price but in smaller quantities. The difference in pricing per tonne between the lowest and highest regional average sales prices was almost a hundred times, showing that region had a profound influence. Some regions also had a mixture of high and low average sales price and yield showing regional variability in pricing which may be explained by varieties produced. A further possibility is the existence of regional upper limits on potential sales price, or that there are diminishing returns in some regions when pursuing higher sales prices or quantity; however these types of relationships may be obfuscated by knowledgeable winegrowers who avoid such pitfalls.
\par
Due to regional differences, different strategies are  employed, such as some regions targeting mass production over higher sales price. This is most notable when grouping regions by climate, especially when considering GI Regions in the 'Hot Very Dry' climate (see Figure~\ref{fig:yield_vs_value_area}). Figure~\ref{fig:spider} also shows that comparatively 'Warm' and 'Dry' Regions manage their resources incredibly efficiently having generally larger areas using similar resources to those in 'Cool' regions but having areas comparable to regions in 'Hot' climates. The coefficients for Model 4 also show a greater benefit of resource use per area when producing grapes at higher average sale prices, showing higher resource use per area reflective of higher average sales price (see Table~\ref{tab:summary}).
%In the preliminary analyses alternative models found that without the direct incorporation of GI Region or year, predictions greatly under performed. The effect of climate in the models was never as significant as the more granular GI regions, and always led to less accurate models. 
Although not chosen over GI region, climate was considered to be a large determinant of the ability to produce larger quantities of grapes, as well as a determinant in grape sale price~\cite{agostaRegionalClimateVariability2012}. The more granular GI Region likely explained a broader mix of geographical phenomenon, such as soil, geology and access to water resources~\cite{abbalDecisionSupportSystem2016,carmonaUseParticipatoryObjectOriented2011}. The interaction between year and GI Region likely accounted for events such as bushfires, which would be impactful, but only at a local level, both in time and space.
\par
We identified two main limitations to our linear modelling. First, model 1 and 2 over-predicting yield may have been due to preventative measures brought on by regional pressures such as fire, frost and disease. More fuel and water was likely used to prevent these issues from spreading within a region, thus disproportionately affecting some vineyards compared to others locally. This type of maintenance is not well captured in the models, especially when considering that some regions, especially those in warmer areas, are not as prone to disease as cooler climates and could potentially have lower fuel and water use per hectare. This could create a discrepancy in vineyards that utilised preventative measures in wetter regions, as opposed to those that did not, thus expending less fuel and energy but risking disease. When reviewing the differences between regions, it is important to consider that vineyards in 'Hot Very Dry' areas can be hundreds of times the size of those in other regions. This limitation could be overcome by incorporating the profitability of vineyards, comparing the financial success of working at different operational scales.
\par
The second limitations was the lack of further explanatory variables to help link models to causal affects. Variables such as the utilisation of renewable energy, contractors, and the occurrence of disease, fire and frost were originally explored to capture the discrepancies between similar vineyards that produced different yields and crop values. However, none of these variables was significantly correlated with the response variables, and did not add to model accuracy, even when considered as interactions. Allowance for nonlinear relationships, specifically through splines, resulted in more normally distributed residuals but at a drastically reduced overall accuracy when comparing $R^2$ and Residual Square Error. Attempts to fully explain small variations was always overshadowed by the dramatic differences in regional trends. Having more data for each region would also be beneficial, allowing greater comparison between regions.
\par
The use of other models such as random forests and decision trees alongside more variables and data may help to uncover the reasons for under or overestimation. These differences could be caused by the use of alternative sustainable practices in the field. Moreover, while there is evidence to suggest that environmentally sustainable practices can reduce costs, and increase efficiency whilst improving the quality of grapes; more research is needed to link these benefits across different regions and climates~\cite{baianoOverviewSustainabilityWine2021,marianiSustainableWinegrowingCurrent2015,montalvo-falconSustainabilityResearchWine2023}.
\section*{Conclusion}

This study delved into the relationships between resource use, grape sales price and yield. The findings underscore the multifaceted nature of vineyard management, where the interplay of size, resource allocation, climate, and regional influences collectively shape both the expected sale price and the quantity of grape yields. Average sales price of grapes was not solely tied to the overall expenditure of resources, but rather to the efficient allocation of resources per area. This emphasises that factors beyond sheer scale contribute significantly to the final sale price of grapes produced. Moreover, regional and yearly variations exhibited substantial effects on vineyard outputs, impacting both sales price and quantity. The connection observed between smaller vineyards and higher grape sale price suggests that the management of smaller properties might be more efficient and profitable.

\nolinenumbers

% \bibliographystyle{plos2015}
\bibliography{references}

%% The Appendices part is started with the command \appendix;
%% appendix sections are then done as normal sections
 \appendix

\section*{Appendix}
\par

\begin{table}[htb]
      \caption{P-values for the non-transformed water used variable's Pearson correlation coefficients.}
      \begin{tabular}{cl}
      Variable                                           & Water Used \\
      Yield                                              & 7.538E-01  \\
      Area                                               & 6.981E-01  \\
      Scope One Emissions                                & 8.883E-01  \\
      $\textrm{Yield}\over\textrm{Area}$                     & 6.836E-01  \\
      Average Price Per Tonne                            & 5.600E-02  \\
      ${\textrm{Average Price per tonne}\over\textrm{Area}}$ & 1.522E-01 
      \end{tabular}
\end{table}


\begin{figure}[htb]
  % \resizebox{\textwidth}{!}{
  % \includegraphics{s1.pdf}}
  \caption{QQ-plot of Model 1.}
\end{figure}

\begin{figure}[htb]
  % \resizebox{\textwidth}{!}{
  % \includegraphics{s2.pdf}}
  \caption{Residuals vs fitted values for Model 1.}
\end{figure}

\begin{figure}[htb]
  % \resizebox{\textwidth}{!}{
  % \includegraphics{s3.pdf}}
  \caption{QQ-plot of Model 2.}
\end{figure}

\begin{figure}[htb]
  % \resizebox{\textwidth}{!}{
  % \includegraphics{s4.pdf}}
  \caption{Residuals vs fitted values for Model 2.}
\end{figure}

\begin{figure}[htb]
  % \resizebox{\textwidth}{!}{
  % \includegraphics{s5.pdf}}
  \caption{QQ-plot of Model 3.}
\end{figure}

\begin{figure}[htb]
  % \resizebox{\textwidth}{!}{
  % \includegraphics{s6.pdf}}
  \caption{Residuals vs fitted values for Model 3.}
\end{figure}

\begin{figure}[htb]
  % \resizebox{\textwidth}{!}{
  % \includegraphics{s7.pdf}}
  \caption{QQ-plot of Model 4.}
\end{figure}

\begin{figure}[htb]
  % \resizebox{\textwidth}{!}{
  % \includegraphics{s8.pdf}}
  \caption{Residuals vs fitted values for Model 4.}
\end{figure}
\begin{figure}[htb]
  % \resizebox{\textwidth}{!}{
  % \includegraphics{s9.pdf}}
  \caption{QQ-plot of Model 5.}
\end{figure}

\begin{figure}[htb]
  % \resizebox{\textwidth}{!}{
  % \includegraphics{s10.pdf}}
  \caption{Residuals vs fitted values for Model 5.}
\end{figure}

% \subsection*{Complete list of coefficients for Model 1.}
% \csvautolongtable[]{model1_coef.csv}

%   \subsection*{Complete list of coefficients for Model 2.}
% \csvautolongtable[]{model2_coef.csv}


%   \subsection*{Complete list of coefficients for Model 3.}
% \csvautolongtable[]{model3_coef.csv}


%   \subsection*{Complete list of coefficients for Model 4.}
% \csvautolongtable[]{model4_coef.csv}


%   \subsection*{Complete list of coefficients for Model 5.}
% \csvautolongtable[]{model5_coefs.csv}

%
%% \section*{}
%% \label{}

%% If you have bibdatabase file and want bibtex to generate the
%% bibitems, please use
%%

%% else use the following coding to input the bibitems directly in the
%% TeX file.


\end{document}
